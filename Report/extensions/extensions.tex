\documentclass[../report.tex]{subfiles}
\begin{document}

An obvious extension to the paper to investigate is to simply change the form of the prior and the variational approximation in an attempt to induce a particular form of latent space. For example a particularly interesting set up would be to define a sparsity inducing prior that encourages each dimension of the latent space to be approximately valued on $\{0, 1\}$. An obvious choice would be a set of sparse Beta distributions (ie. ones in which the shape parameters $\alpha, \beta < 1$), but one could also use pairs of univariate Gaussians with means $0$ and $1$ and small variances.

Such a prior would be useful for two reasons - firstly it would allow one to provide a binary encoding for a data set by truncating the posterior approximation for any particular observation to be exactly vector binary valued allowing for a large amount of lossy compression. The posterior distribution over the parameters $\theta$ and latent values $\mathbf{z}_i$ also contains rotational symmetry which may affect the quality of the approximate inference if it attempts to place posterior mass over the entirety of this. Were a prior such as the one proposed used, this rotational symmetry would be destroyed and replaced with a ``permutation symmetry'', similar to that found in a finite mixture model.

\end{document}