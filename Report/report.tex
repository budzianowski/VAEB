\documentclass[10pt,a4paper]{article}
%twocolumn
\usepackage[utf8]{inputenc}
\usepackage[T1]{fontenc}
\usepackage{amsmath}
\usepackage{amsfonts}
\usepackage{amssymb}
\usepackage{lmodern}
\usepackage{subfiles}
\newcounter{defcounter}
\setcounter{defcounter}{0}

%%% PAGE DIMENSIONS
\usepackage{geometry} % to change the page dimensions
\geometry{a4paper} % or letterpaper (US) or a5paper or....
\geometry{margin=2cm} % for example, change the margins to 2 inches all round

\usepackage[parfill]{parskip} % Activate to begin paragraphs with an empty line rather than an indenet
\usepackage{graphicx}

\usepackage{xcolor}
\usepackage{listings}
\definecolor{codegreen}{rgb}{0,0.6,0}
\definecolor{codegray}{rgb}{0.5,0.5,0.5}
\definecolor{codepurple}{rgb}{0.58,0,0.82}
\definecolor{backcolour}{rgb}{0.95,0.95,0.92}
\lstset{basicstyle=\ttfamily,
 backgroundcolor=\color{backcolour},   
  showstringspaces=false,
  commentstyle=\color{red},
  keywordstyle=\color{blue}
}

%getting the dots
\usepackage{tocloft}
\makeatletter
\renewcommand{\@seccntformat}[1]{\csname the#1\endcsname.\quad}
\makeatother

\renewcommand{\cftsecleader}{\cftdotfill{\cftdotsep}}

%%%Definitions
\newcommand{\bet}{\boldsymbol{\beta}}
\newcommand{\xx}{\widetilde{\boldsymbol{x}}^{(n)}}

\usepackage{float}
\usepackage{multirow}

% no linebreaks after subsection!
\usepackage{titlesec}
%\titleformat{\subsection}[runin]{\normalfont\large\bfseries}{\thesubsection}{1em}{}

\usepackage{float}
\usepackage{multirow}

\begin{document}
\begin{center}
\Large{\textsc{
University of Cambridge}}\\
\vspace{.5cm}
\large\textbf{MLSALT 4}\\
\vspace{.5cm}
\large{Tom}\\
\large{Will Tebbutt, wct23, Darwin College}\\
\large{Paweł Budzianowski, pfb30, Clare Hall}\\
\end{center} 

\section{Introduction}
\subfile{introduction/introduction.tex}

\section{Stochastic Variational Inference}
\subfile{svi/svi.tex}

\section{The Variational Autoencoder}
\subfile{vae/vae.tex}

\section{Reconstruction}
\subfile{recon/recon.tex}

\section{Full variantional bayes}
\subfile{fvb/fvb.tex}

\begin{lstlisting}[ basicstyle=\small,language = python , breaklines=true]=


\end{lstlisting}



 {\small \textbf{Table 1.} The results of baseline and focus tracker on DSTC data.}
\begin{center}
\begin{tabular}{c | c c c} \label{tab1}
Baseline &  Joint Goals & Requested & Method \\ \hline 
Accuracy & 0.569 & 0.914 & 0.682\\
L2 norm & 0.834  & 0.120& 0.576\\
\end{tabular} 
\end{center}

\begin{figure}[!htb]
\minipage{0.5\linewidth}%
% \includegraphics[width=\linewidth]{mcc1}
\endminipage 
\minipage{0.5\linewidth}  
%\includegraphics[width=\linewidth]{mcc2}
\endminipage\hfill
  \caption[1]{Evaluations results on DSTC task with different specification parameter.}
\end{figure}

\section{Conclusion}
\subfile{conclusion/conclusion.tex}

\begin{thebibliography}{15}
\bibitem{sutton} R.S. Sutton, A. G. Barto, \emph{Reinforcement Learning: An Introduction}, The MIT Press (March 1998), 2011.


\end{thebibliography}

\end{document}

