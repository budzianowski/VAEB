%%%%%%%%%%%%%%%%%%%%%%%%%%%%%%%%%%%%%%%%%%%
%
% From a template maintained at https://github.com/jamesrobertlloyd/cbl-tikz-poster
%
% Code near the top should be fairly standard and not need to be changed
%  - except for the document class
% Code lower down is more likely to be customised
%
%%%%%%%%%%%%%%%%%%%%%%%%%%%%%%%%%%%%%%%%%%%


\documentclass[landscape,a0b,final,a4resizeable]{include/a0poster}

\usepackage{multicol}
\usepackage{color}
\usepackage{morefloats}
\usepackage[pdftex]{graphicx}
\usepackage{rotating}
\usepackage{amsmath, amsthm, amssymb, bm}
\usepackage{array}
\usepackage{booktabs}
\usepackage{multirow}
\usepackage{hyperref}


\usepackage{include/picins}
\usepackage{tikz}
\usetikzlibrary{shapes.geometric,arrows,chains,matrix,positioning,scopes,calc}
\tikzstyle{mybox} = [draw=white, rectangle]
\definecolor{darkblue}{rgb}{0,0.08,0.45}
\definecolor{blue}{rgb}{0,0,1}

\usepackage{dsfont}

\input{include/jlposter.tex}

\input{include/preamble.sty}
\newcommand{\vv}{\mathbf{v}}

\begin{document}
\begin{poster} 

% Potentially add some space at the top of the poster
\vspace{0\baselineskip}


%%% Header
\begin{center}
\begin{pcolumn}{0.99}

\newcommand{\logowidth}{0.11\textwidth}

\pbox{0.99\textwidth}{}{linewidth=2mm,framearc=0.3,linecolor=camdarkblue,fillstyle=gradient,gradangle=0,gradbegin=white,gradend=white,gradmidpoint=1.0,framesep=1em}{
%
%%% Cambridge Logo
\begin{minipage}[c]{\logowidth}
  \begin{center}
    \includegraphics[width=10cm]{badges/cambridgecrest}
  \end{center}
\end{minipage}
%
%%% Title
\begin{minipage}[c][9cm][c]{0.76\textwidth}
  \begin{center}
    {\sffamily \VeryHuge \textbf{Convolutional Networks on Graphs\\[4mm] for Learning Molecular Fingerprints}}\\[10mm]
    {\huge\sffamily \Huge David Duvenaud*, Dougal Maclaurin*, Jorge Aguilera-Iparraguirre \\[7.5mm]
    %\texttt{\{ti242, dkd23, zoubin\}@cam.ac.uk}
    }
  \end{center}
\end{minipage}
%
%
% Harvard logo
\begin{minipage}[c]{\logowidth}
  \begin{flushright}
    \includegraphics[width=15cm, clip]{badges/camtext}
  \end{flushright}
\end{minipage}
%
}
\end{pcolumn}
\end{center}

\vspace*{3cm}

\Large


%%%%%%%%%%%%%%%%%%%%%%%%%%%%%%%%%%%%%%%%%%%%%%%%%%%%%%%%%%%%%%%%%%%%%%
%%% Beginning of Document
%%%%%%%%%%%%%%%%%%%%%%%%%%%%%%%%%%%%%%%%%%%%%%%%%%%%%%%%%%%%%%%%%%%%%%


\begin{multicols}{3}

\mysection{How to do regression on graphs?}

%\vspace{-0.5in}

\begin{tabular}{cc}
\begin{minipage}[c]{0.45\columnwidth}
\begin{itemize}
  \item Input can be any size or shape
  \item Hard to turn into fixed-length vector
  \item In our case, graphs represent molecules
  \item Applications to photovoltaics, organic LEDS, flow batteries and pharmaceuticals
\end{itemize}
\end{minipage} & 
\begin{minipage}[c]{0.55\columnwidth}
%\includegraphics[width=\columnwidth]{../talks/talkfigs/learning_curves_3.pdf}
\centerline{\includegraphics[width=1.0\columnwidth, clip, trim=4mm 0mm 4mm 4mm]{figures/how-fingerprints.png}}
\end{minipage}
\end{tabular}

%\vspace{-0.5in}

\vspace{1.5in}

\mysection{Circular fingerprints}

%Also called Morgan fingerprints, or ECFP

\vspace{0.5in}

\begin{tabular}{cc}
\begin{minipage}[c]{0.5\columnwidth}
\begin{itemize}
  \item Maps variable-sized molecular graph to fixed-length binary vector
  %\item Does this by hashing self with neighbors iteratively
  \item Binary features indicate presence of substructures
\end{itemize}

\vspace{0.5in}

Can be efficiently computed using local operations:

\begin{itemize}
  \item At each layer, hash the features of each atom and its neighbors/bonds
  \item More layers correspond to increasing radius of substructures
  \item Interpret each hash as integer and set that entry to one
\end{itemize}
\end{minipage} & 
\begin{minipage}[c]{0.5\columnwidth}
%\includegraphics[width=\columnwidth]{../talks/talkfigs/learning_curves_3.pdf}
\centerline{\includegraphics[width=0.9\columnwidth, clip, trim=4mm 12mm 4mm 4mm]{figures/fig_1}}
\end{minipage}
\end{tabular}

\vspace{0.5in}

Was state-of-the-art for large-scale regression and classification.



\newpage %%%%%%%%%%%%%%%%%%%%%%%%%%%%%%%%%%%%%%%%%%%%%%%%%%%%%%%%%%%%%%%%%%%%%%%%%%%%%%%%%%%%%%%%%%%%%%%


\mysection{Learning Curves - la vs lb}

Comparison of learning

\vspace{0.5em}

\begin{center}
\includegraphics[width=.5\columnwidth]{figures/fig_2.pdf}
\end{center}

\vspace{0.5em}

\mysection{Visualisation of learned manifolds}

Comparison of learning

\vspace{0.5em}

\begin{center}
\includegraphics[width=.5\columnwidth]{figures/fig_2.pdf}
\end{center}

\vspace{0.5em}




\newpage %%%%%%%%%%%%%%%%%%%%%%%%%%%%%%%%%%%%%%%%%%%%%%%%%%%%%%%%%%%%%%%%%%%%%%%%%%%%%%%%%%%%%%%%%%%%%%%


\mysection{Full VB}

Large random weights give similar behavior to circular fingeprints:
\vspace{0.5em}

\begin{tabular}{cc}
\begin{minipage}[c]{0.48\columnwidth}
\includegraphics[width=\columnwidth]{figures/fig_2.pdf}
\end{minipage} & 
\begin{minipage}[c]{0.48\columnwidth}
\begin{center}
\vspace{0.5cm}\includegraphics[width=\columnwidth]{figures/fig_3.pdf}
\end{center}
\end{minipage}
\end{tabular}

\vspace{0.5em}

Small random weights already much better than circular fingerprints!

Can do even better by optimizing for given task.

\vspace{1in}

\mysection{Architecture experiments}

We tested various set-up of the autoencoder

\begin{tabular}{cc}
\begin{minipage}[c]{0.25\columnwidth}
\begin{itemize}
\item Increasing the depth of the encoder
\end{itemize}
\end{minipage} & 
\begin{minipage}[c]{0.75\columnwidth}
%\includegraphics[width=1.0\columnwidth, clip, trim=4mm 0mm 4mm 4mm]{figures/how-fingerprints.png}
\end{minipage}
\end{tabular}


\begin{tabular}{cc}
\begin{minipage}[c]{0.25\columnwidth}
\begin{itemize}
\item Different activations functions
\end{itemize}
\end{minipage} & 
\begin{minipage}[c]{0.75\columnwidth}
%\includegraphics[width=1.0\columnwidth, clip, trim=4mm 0mm 4mm 4mm]{figures/how-fingerprints.png}
\end{minipage}
\end{tabular}


\begin{tabular}{cc}
\begin{minipage}[c]{0.25\columnwidth}
\begin{itemize}
\item Different activations functions
\end{itemize}
\end{minipage} & 
\begin{minipage}[c]{0.75\columnwidth}
\includegraphics[width=1.0\columnwidth, clip, trim=4mm 0mm 4mm 4mm]{figures/mnist_LAvsLB}
\end{minipage}
\end{tabular}
\mysection{Conclusion}

\begin{itemize}
\item Can learn graph features end-to-end!
\item Works on other types of graphs too 
\item Code at \url{github.com/HIPS/neural-fingerprint}
\item Autodiff package that works on standard Numpy code:\\ \url{github.com/HIPS/autograd}
\end{itemize}



\end{multicols}
\end{poster}

\end{document}

